%%%% ijcai23.tex

\typeout{IJCAI--23 Instructions for Authors}

% These are the instructions for authors for IJCAI-23.

\documentclass{article}
\pdfpagewidth=8.5in
\pdfpageheight=11in

% The file ijcai23.sty is a copy from ijcai22.sty
% The file ijcai22.sty is NOT the same as previous years'
\usepackage{ijcai23}

% Use the postscript times font!
\usepackage{times}
\usepackage{soul}
\usepackage{url}
\usepackage[hidelinks]{hyperref}
\usepackage[utf8]{inputenc}
\usepackage[small]{caption}
\usepackage{graphicx}
\usepackage{amsmath}
\usepackage{amsthm}
\usepackage{booktabs}
\usepackage{algorithm}
\usepackage{algorithmic}
\usepackage[switch]{lineno}

% Comment out this line in the camera-ready submission
\linenumbers

\urlstyle{same}

% the following package is optional:

% See https://www.overleaf.com/learn/latex/theorems_and_proofs
% for a nice explanation of how to define new theorems, but keep
% in mind that the amsthm package is already included in this
% template and that you must *not* alter the styling.
\newtheorem{example}{Example}
\newtheorem{theorem}{Theorem}

% Following comment is from ijcai97-submit.tex:
% The preparation of these files was supported by Schlumberger Palo Alto
% Research, AT\&T Bell Laboratories, and Morgan Kaufmann Publishers.
% Shirley Jowell, of Morgan Kaufmann Publishers, and Peter F.
% Patel-Schneider, of AT\&T Bell Laboratories collaborated on their
% preparation.

% These instructions can be modified and used in other conferences as long
% as credit to the authors and supporting agencies is retained, this notice
% is not changed, and further modification or reuse is not restricted.
% Neither Shirley Jowell nor Peter F. Patel-Schneider can be listed as
% contacts for providing assistance without their prior permission.

% To use for other conferences, change references to files and the
% conference appropriate and use other authors, contacts, publishers, and
% organizations.
% Also change the deadline and address for returning papers and the length and
% page charge instructions.
% Put where the files are available in the appropriate places.


% PDF Info Is REQUIRED.
% Please **do not** include Title and Author information
\pdfinfo{
/TemplateVersion (IJCAI.2023.0)
}

\title{Database-driven Chatbot}


% Single author syntax
\author{
    Eduardo Neves
    \affiliations
    Universidade de Coimbra
    \emails
    email@example.com
}

% Multiple author syntax (remove the single-author syntax above and the \iffalse ... \fi here)
\iffalse
\author{
First Author$^1$
\and
Second Author$^2$\and
Third Author$^{2,3}$\And
Fourth Author$^4$
\affiliations
$^1$First Affiliation\\
$^2$Second Affiliation\\
$^3$Third Affiliation\\
$^4$Fourth Affiliation
\emails
\{first, second\}@example.com,
third@other.example.com,
fourth@example.com
}
\fi

\begin{document}

\maketitle

\begin{abstract}
    With the rising implementation of chatbots in today's technologies
    (!!!) their implementations has risen in various eneterprises. One
    example is the inclusion of many chatbots in websites. With many 
    applications 
\end{abstract}

\section{Introdução}
A Chatbot is 


\section{Dados e Abordagem}
The data for a typical Chatbot is based on real or made-up conversations between two parts. In reality, "any given tet with utility" can be used to build the algorithm. Para este trabalho foi usado um 'dataset' baseado em em legendas de filmes para português.

\subsection{Conjunto de dados}
O pacote de dados usado foi tirado do projeto "OPUS ... the open parallel corpus", uma coleção de testos traduzidos da internet baseada em produtos 'open source'. Este conjunto estava inserido na ala 'Open Subtitles', retirada opensubtitles.org(!!!!!).

Como o foco do projeto é um algoritmo em portguês, retirou-se o dataset "pt", que coleciona mais de 500000 legendas de filmes até o ano de 2017. Separados por anos, alguns filmes contam com várias versões do mesmo filme. O seu tratamento é explicado na secção seguinte. 





\section{Experimentação e Metodologia}

\subsection{Tratamento dos dados}
Com o conjunto de dados inicial com redundâncias e pouco estruturado, procurou-se correr um pequena organização à informação. A organização do fluxo de entrada para treino é feita aquando a inicialização do processo.

\subsubsection{Redução de redundâncias}
Como referido, algumas versões do mesmo filme são dispostas e organizadas em conjunto. Ao remover este entrave, pode-se remover uma camada na diretoria e agrupar apenas por ano. Apesar desta diferenciação não ser necessária poderá ser útil em comparações ortográficas ou até numa maior confiança de traduções mais recentes. Para selecionar o ficheiro mais relevante apenas se retirou o ficheiro com menor volume de espaço no disco para 'performance'.

\subsubsection{Alteração do formato}
O \textit{dataset} original continha apenas ficheiros em .xml, com muitas "especificações!!!!!" desnecessárias a este trabalho. Um exemplo é a separação por palavras das falas de cada personagem. Simplificou-se então para um formato .json, onde cada entrada é a fala de uma personagem.

\subsubsection{Redução de amostragem}
Para além da redução na alteração do formato, reduziu-se também a quantidade de dados usados para o programa. Utilizou-se apenas filmes a partir de 2000, o que resultou em 37851 filmes, com uma média de 455 falas de personagens.

\subsection{Modelo}

\subsubsection{lstm}




\section{Conclusions}

%% The file named.bst is a bibliography style file for BibTeX 0.99c
\bibliographystyle{named}
\bibliography{ijcai23}

\end{document}

